\documentclass{article}

\usepackage[utf8]{inputenc}
\usepackage[T1]{fontenc}
\usepackage{hyperref}
\usepackage{tabularx}
\usepackage{array}
\usepackage{fancyhdr}
\usepackage{graphicx}
\usepackage[a4paper]{geometry}
\usepackage{multicol}

\title{Rapport de l'API REST \\ Zoo Manager}
\author{par Geoffrey Spaur}
\date{15 mars 2017}
\pagestyle{fancy}
\lhead{Rapport de l'API REST - Zoo Manager \\ \textbf{M1GIL} - Geoffrey Spaur}
\rhead{\includegraphics[scale=0.5]{logo_univ_rouen.png}}
\setlength{\headsep}{1cm}
\begin{document}

\maketitle
\newpage
\tableofcontents{}
\newpage
\section{Présentation}
	Le projet a pour but de réaliser une API REST. Ce projet a pour thème la gestion d'un zoo. Nous devrons donc réaliser des fonction d'ajout, de modification ou de suppression. Ainsi que des fonctions de recherche.
	
\section{Problèmes Rencontrés}

\subsection{Transmition des positions}
\subsection{Création de cages}
\begin
L'ajout du scénario dans la seconde partie du TP m'a permis de constater que certaine fonction était inutile ou qu'il en manquait. Suite à cette partie, j'ai donc pris l'initiative d'ajouter ou de ne pas faire certaine fonction.\end
Parmi les fonctions ajoutées, nous avons notamment une fonction permettant de créer une cage. En revanche, une cage ne peux jamais être détruite.

\subsection{Création des résidents}
\subsection{URLEncoder}
  
\end{document}