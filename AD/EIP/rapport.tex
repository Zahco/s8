\documentclass{article}

\usepackage[utf8]{inputenc}
\usepackage[T1]{fontenc}
\usepackage{hyperref}
\usepackage{tabularx}
\usepackage{array}
\usepackage{fancyhdr}
\usepackage{graphicx}
\usepackage[a4paper]{geometry}
\usepackage{multicol}
\usepackage{listings}

\title{Architecture distribuée \\ EIP}
\author{par Geoffrey Spaur}
\date{8 Mai 2017}
\pagestyle{fancy}
\lhead{Architecture distribuée - EIP \\ \textbf{M1GIL} - Geoffrey Spaur}
\rhead{\includegraphics[scale=0.5]{logo_univ_rouen.png}}
\setlength{\headsep}{1cm}
\begin{document}

\maketitle
\newpage
\tableofcontents{}
\newpage
\section{Présentation}
  Le projet a pour but de présenter les EIP afin d'intégrer les services web.
\section{Problèmes Rencontrés}
  \subsection{JSON}
    Le service Zoo Manager utilisé répondait sous format \emph{JSON}. 
    Or avec la configuration de base que nous avions, notre EIP ne comprenait que le \emph{XML}.
    Nous avons donc du rajouter une dépendance à notre projet:
    \begin{lstlisting}[language=xml]
  <!-- https://mvnrepository.com/artifact/org.apache.camel/camel-jackson -->
    <dependency>
	<groupId>org.apache.camel</groupId>
	<artifactId>camel-jackson</artifactId>
	<version>2.18.3</version>
    </dependency>
    <dependency>
	<groupId>org.apache.camel</groupId>
	<artifactId>camel-xstream</artifactId>
	<version>2.9.2</version>
    </dependency>
    \end{lstlisting}
    Pour pourvoir utiliser le \emph{Unmarshalling}:
    \begin{lstlisting}[language=java]
  ...
    .to("http://127.0.0.1:8080/animals/findByName/")
    .unmarshal().json(JsonLibrary.Jackson)
    .log("${body}")
  ...
    \end{lstlisting}
  \subsection{Multi instances}
    Pour pouvoir instancier plusieurs serveurs, j'ai du changer les ports de ces dernier pour éviter les conflits:
    \begin{lstlisting}[language=java]
  System.getProperties().put( "server.port", "8081");
    \end{lstlisting}
    Afin d'éviter l'agrégation suggéré dans le sujet, j'ai travaillé sur les exceptions.
    Ainsi, si un animal, n'était pas trouvé, notre \emph{EIP} demander au service suivant jusqu'à le trouver.
    \begin{lstlisting}[language=java]
  .doTry()
    .to("http://127.0.0.1:8080/animals/findByName/")
    .log("${body}")
  .doCatch(HttpOperationFailedException.class)
      .doTry()
	  .to("http://127.0.0.1:8081/animals/findByName/")
	  .log("${body}")
      .doCatch(HttpOperationFailedException.class)
	  .log("Animal introuvable")
      .end()
  .end()
    \end{lstlisting}


\end{document} 
